\documentclass[10pt]{article}
\usepackage{amsmath}
\usepackage{amssymb}
\usepackage[ruled]{algorithm2e}
\newtheorem{theorem}{Theorem}[section]
\newtheorem{corollary}{Corollary}[theorem]
\newtheorem{lemma}[theorem]{Lemma}
\newcommand{\Not}[1]{$\neg #1$}

\usepackage[a4paper, total={6in, 8in}]{geometry}
\begin{document}

\section{Definitions and Notations}

proved by Chvatal and Reed~\cite{2satratio}.
\begin{theorem}\label{lem:chvatal}
    For any random 2-SAT formula F with $N$ variables and $\alpha N$ clauses, where $N$ tends to infinity, F is satisfiable with probability $1 - o(1)$ for $\alpha < 1$, and F is unsatisfiable with probability $1 - o(1)$ for $\alpha > 1$.
\end{theorem}
\section{Local Heuristics}
\par
% basis for 
First of all, we are going to start by presenting the most common and yet the simpliest local heuristics, namely Pure Literal (PL) and Unit Clause (UC). Their behaviour can be described in terms of free and forced steps~\cite{satdiffeq}. More precisely, forced steps are those decisions that cannot go wrong, that is we can satisfy a certain variable, while preserving the satisfiability of the formula. On the other hand, free steps are those decisions that do not have these guarantees.
%In the following sections, we will restrict ourserlves to the formulations of PL and UC, where in the free steps we choose a literal at random and satisfy it.

\par
The forced steps taken by the Pure Literal and Unit Clause are universally good, and that is why the combination of these two heuristics is a basic ingredient of much more elaborate heuristics. However, the more advanced make a better use of the free steps, which significantly boosts their performance, but makes them much more difficult to analyse.

\subsection{Pure Literal}
\par
First, let's call a literal $l$ \textit{pure}, if there is no clause containing $\neg l$. Observe that if given CNF $\phi$ contains a pure literal $l$, then we can satisfy $l$ in $\phi$ without changing the satisfiability of $\phi$, because we delete all occurances of $var(l)$ from $\phi$, which results in a less constrained formula.
This observation is the driving factor behind the Pure Literal heuristic, which is presented in Algorithm~\ref{alg:pl}. 
\par
%seems raw
Broder et al.~\cite{pureliteral} showed that the Pure literal almost surely succeeds in finding a satisfying assignment when the ratio of clauses to variables is $\alpha \le 1.63$. 
\begin{algorithm}
\caption{Pure Literal Heuristic}\label{alg:pl}
\KwIn{A k-SAT formula $\phi$}
\KwOut{\textit{True} if $\phi$ is satisfiable, \textit{False} otherwise}
\While{$Vars(\phi) \neq \emptyset$}{
    \eIf{exists pure literal $l$}{
        \tcp{Forced Step}
        satisfy $l$ in $\phi$\; 
    }{
        \tcp{Free Step}
        $x \gets$ randomly chosen variable in $\phi$ \;
        Choose uniformly at random $l \in \{x, \neg x\}$ and satisfy it in $\phi$\; 
    }
}
\eIf{$\phi$ contains empty clause}{
    \Return{\textit{False}}
}{
    \Return{\textit{True}}
}
\end{algorithm}
\par
The above algorithm runs in time $O(N k \alpha)$, as the existence of pure literal can be checked in $O(1)$ by maintaining for each variable $x$ the number of clauses containing $x$ and $\neg x$. Moreoever, pruning $\phi$ after satysfying a variable, amortizes to $O(k \alpha)$ per iteration.


\subsection{Unit Clause}
Simirarly to PL, UC exploits a very simple observation. Namely, if any clause $C$ contains only one literal $l$, then any satisfying assignment to $\phi$ must satisfy $l$, otherwise it would contain an empty clause, yielding an unsatisfiable formula. The algorithm for UC is presented below:
\begin{algorithm}
\caption{Unit Clause Heuristic}\label{alg:uc}
\KwIn{A k-SAT formula $\phi$}
\KwOut{\textit{True} if $\phi$ is satisfiable, \textit{False} otherwise}
\While{$Vars(\phi) \neq \emptyset$}{
    \eIf{$\phi$ contains unit clause $\{l\}$}{
        \tcp{Forced Step}
        satisfy $l$ in $\phi$\; 
    }{
        \tcp{Free Step}
        $x \gets$ randomly chosen variable in $\phi$ \;
        choose uniformly at random $l \in \{x, \neg x\}$ and satisfy it in $\phi$\; 
    }
}
\eIf{$\phi$ contains empty clause}{
    \Return{\textit{False}}
}{
    \Return{\textit{True}}
}
\end{algorithm}
\par Similarly to PL, UC runs in time $O(N k \alpha)$, beacause after satisfying a varaible $x$, the only new clauses are the ones that previously contained $x$. Therefore, the detection of new unit clauses amortizes to $O(k \alpha)$.
\subsubsection*{Analysis}
%framework for heuristics that can be described as the card game
%analysis of UC on 4-SAT

Despite the simplicity of the UC heuristic, the analysis of its performance on random k-SAT formulas is far from trivial. Achlioptas~\cite{satdiffeq} presents a very elegant framework for analysing the performance of a family of heuristics that includes UC\@. In the following section, we will present the framework by applying it to the UC heuristic on a random 4-SAT formula.
\par
First, let's start with new notation. Let $C_i(t)$ denote the number of clauses in $\phi$ that contain $i$ literals after $t$ steps of UC algorithm. Similarly, let $V(t)$ denote the set of variables present in $\phi$ after $t$ steps. Then, we can present a crucial lemma proved by Achlioptas~\cite{satdiffeq} that is the centre of the whole framework:
\begin{lemma}\label{lem:uni_rand}
    For every $ 0 \le t \le n$, any clause after $t$ steps contains a uniformly random set of i distinct non-complementary literals from $L(V(t))$.
\end{lemma}

First, let's start with the observation that the algorithm fails only on forced steps, more precisely when $\phi$ contains $\{ l \}$ and $\{ \neg l \}$. One way, to guarantee that the algorithm does not encounter such a situation, is to assert that on each iteration the
expected number of unit clauses generated is less than 1. This way, in expectation, at each step there is at most one unit clause, hence a contradiction cannot be derived.
Formally, if $\mathbb{E}[\Delta C_1(t)] = C_2(t)/(n - t) < 1 - \gamma $, then w.h.p. UC succeeds
\par 
Analysing $C_2(t)$ on its own is a hard task. However, Achlioptas~\cite{satdiffeq} proposes a clever idea of modelling $\Delta C_2(t)$ instead, together with the change in clauses of size 3, $\Delta C_3(t)$, and of clauses of size 4, $\Delta C_4(t)$. Here, if we condition on $C_2(t), C_3(t), C_4(t)$, then by Lemma~\ref{lem:uni_rand} we get that:

\begin{align}\begin{split}
    \Delta C_4(t) &= - A
\\
    \Delta C_3(t) &= B - C
\\
    \Delta C_2(t) &= D - E
\end{split} \end{align}
$where$
\begin{align}
\begin{split}
    A &= Bin(C_4(t), \frac{4}{n - t})\\
    B &= Bin(C_4(t), \frac{2}{n - t})\\
    C &= Bin(C_3(t), \frac{3}{n - t})\\
    D &= Bin(C_3(t), \frac{3}{2(n - t)})\\
    E &= Bin(C_2(t), \frac{2}{n-t})
\end{split}
\end{align}

In order to give a little bit of intuition behind the above equations, let's consider the case of $\Delta C_3(t)$. Suppose, we are satisfying literal $l$, then $C_3(t)$ is increased by the number of clauses which previously were of size 4 and contained \Not{l}, and decreased by the number of clauses of size 3 that contained $l$ or \Not{l}. Note that the probability of the former is $\frac{2}{n - t}$, and the probability of the latter is $\frac{3}{(n - t)}$, yielding the equation for $\Delta C_3(t)$. The same reasoning applies to the other cases.
\par
Now, we observe 2 things:
\begin{itemize}
    \item The probability that $\Delta C_i(t)$ deviates from its expected value is small, due to the properties of Binomial distribution.
    \item $\Delta C_2(t)$ is smooth in $t$ and $C_i(t)$ (k-Lipschitz smooth), that is small change in either $t$ or $C_i(t)$ does not change significantly the value of $\Delta C_i(t)$
\end{itemize}
This observations hints that the values $C_i(t)$ do not diverge significantly from their mean trajectories, which can be rigorously proven using Wormland's Theorem. To do so, we consider $C_i(t)$ in a continuous setting, by defining $c_i(x) = C_i(xn) / n$ for $x \in [0, 1] $. By doing so, $\Delta C_i(t)$ translates to $\frac{d}{dx} c_i(x)$, yielding a system of differential equations describing the evolution of $c_i(x)$. Moreover, here the second observation of smoothness of $\Delta C_i(t)$ can be formalized, i.e.\ we require that $\frac{d}{dx} c_i(x)$ is $k$-Lipschitz smooth, for some constant $k$. Having done that, we can finally apply Wormland theorem, that states that for any $\epsilon > 0$ the solution to the system of differential yields a trajectory of $c_i(x)$, that w.h.p.\  closely approximates the actual values of a random process for $x \in [0, 1 - \epsilon]$.
\par
This way, we obtain following equations for $C_i(t)$ when $t \leq (1- \epsilon)n$:
\begin{align}
\begin{split}
    C_2(t) &= \frac{3}{2} \alpha {(\frac{t}{n})}^2{(1 - \frac{t}{n})}^2 + o(n)
    \\
    C_3(t) &= 2 \alpha {(1 - \frac{t}{n})}^3 \frac{t}{n} + o(n)
    \\
    C_4(t) &= \alpha {(1 - \frac{t}{n})}^4 + o(n)
\end{split}
\end{align}
Recall, that our fomrula is w.h.p satisfiable if $C_2(t)/(n - t) < 1 - \gamma $, which implies if $\alpha < 4.5$, then UC within its first $n(1 - \epsilon)$ iterations, will not encounter a contradiction. However, Wormland theorem does not tell us anything about $C_i(t)$ when $t > (1 - \epsilon) n$. But after setting $\epsilon = 0.1$ and $t_e = \lfloor (1 - \epsilon)n \rfloor$, we get that $C_2(t_e) + C_3(t_e) + C_4(t_e) < \frac{2}{3} (n - t_e)$. Therefore, after $t_e$ steps, if we erase from every clause of size 3 a randomly chosen variable, and from every clause of size 4 a randomly chosen pair of variables, then we will obtain a random 2-CNF formula with ratio of clauses to variables less than $\frac{2}{3}$. This way, we can apply Theorem~\ref{lem:chvatal} to conclude that w.h.p.\ the formula is satisfiable.
\par
On the other hand, if $\alpha > 4.5$, then $C_2(t)/(n-t) > 1 + \gamma$, then again by Theorem~\ref{lem:chvatal} we conclude that w.h.p.\ the formula is unsatisfiable.







\bibliographystyle{plain}
\bibliography{../refs}{}

\end{document}


